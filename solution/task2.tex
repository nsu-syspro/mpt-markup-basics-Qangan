\documentclass[a4paper, 12pt]{article}

\usepackage{cmap}
\usepackage[T2A]{fontenc}
\usepackage[utf8]{inputenc}
\usepackage[english, russian]{babel}

\begin{document}
\section{Условие}
Решите уравнение:
\[x^2 - 5x + 6 = 0\]
\section{Решение}
Для решения данного квадратного уравнения воспользуемся формулой:
\[x = \frac{-b \pm \sqrt{D}}{2a},\]
где \( a = 1 \), \( b = -5 \), \( c = 6 \).
Вычислим дискриминант по формуле:
\[D = b^2 - 4ac = (-5)^2 - 4 \cdot 1 \cdot 6 = 25 - 24 = 1\]
Теперь подставим значения:
\[x = \frac{-(-5) \pm \sqrt{1}}{2 \cdot 1} = \frac{5 \pm 1}{2}\]
Теперь найдём корни:
\[x_1 = \frac{5 + 1}{2} = \frac{6}{2} = 3\]
\[x_2 = \frac{5 - 1}{2} = \frac{4}{2} = 2\]
Таким образом, корнями уравнения \( x^2 - 5x + 6 = 0 \) будут:
\[x_1 = 3, \quad x_2 = 2\]
\end{document}
